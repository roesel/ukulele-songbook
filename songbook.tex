\documentclass[twocolumn]{article}

% Setting page margins
\usepackage[margin=0.5in]{geometry}

% ----------------------------------------------------------------------------------------
\usepackage[utf8]{inputenc}
\usepackage[english,czech]{babel}
\usepackage{csquotes}

\makeatletter
\adddialect\l@CZECH\l@czech
\makeatother

\usepackage{textcomp}
\usepackage[T1]{fontenc}
\usepackage{lmodern}

% ----------------------------------------------------------------------------------------

% Ukulele Chords section
\usepackage{etoolbox}
\usepackage{tikz}
\newcommand\drawukulelechord[2][]{%
	\edef\chord{#2}%
	\begin{tikzpicture}[x=1.3ex,y=1.3ex,line cap=round,line width=.4pt,
	baseline=(current bounding box.center),#1]
	\draw[line width=.6pt] (1,0) -- (4,0);
	\foreach \d in {1,...,4}{\draw (1,-\d) -- (4,-\d);}
	\foreach \d[count=\p] in \chord {
		\draw (\p,0) -- (\p,-4.5);
		\ifdefstring{\d}{-1}{
			\draw (\p,.25) +(-.125,-.125) -- +(.125,.125)
			+(-.125,.125) -- +(.125,-.125);
		}{
			\ifdefstring{\d}{0}{
				\draw (\p,.25) circle(.125); % opened string
			}{
				\fill (\p,.5-1*\d) circle(.25);
			}
		}
	}
	\path[use as bounding box] (0.5,.5) rectangle (4.5,-5);
	\end{tikzpicture}%
}
\makeatletter
\newcommand\defineukulelechord[2]{%
	\csdef{@ukulelechord@#1}{\drawukulelechord{#2}}%
}
\newcommand\ukulelechord[1]{%
	\ifcsdef{@ukulelechord@#1}{%
		\csuse{@ukulelechord@#1}%
	}{%
		\GenericError{}{Undefined ukulele chord '#1'}{}{}% 
	}%
}
\makeatother

\defineukulelechord{A maj}{2,1,0,0}
\defineukulelechord{A 6}{2,1,2,0}
\defineukulelechord{G sharp maj}{5,3,4,3}
\defineukulelechord{A flat maj}{5,3,4,3}
\defineukulelechord{G sharp 6}{1,3,1,3}

\defineukulelechord{Am}{2,0,0,0}
\defineukulelechord{C}{0,0,0,3}
\defineukulelechord{G}{0,2,3,2}
\defineukulelechord{Em}{0,4,3,2}
\defineukulelechord{D}{2,2,2,0}

\defineukulelechord{Fmaj7}{2,4,1,3}

\defineukulelechord{D/F+}{0,2,5,2} % needs checking
\defineukulelechord{A}{2,1,0,0} % needs checking
\defineukulelechord{E}{1,4,0,2} % needs checking
\defineukulelechord{Ami}{2,0,0,0} % needs checking

% ----------------------------------------------------------------------------------------

\newcommand{\hint}[1]
{
	\StrSubstitute{#1}{+}{\#}[\temp]%
	\begin{tabular}{@{}rcc@{}}
		& \small{\temp} \\
		& \ukulelechord{#1}  \\	
	\end{tabular}
}

% Chordbox
\usepackage{menukeys}

% create a new simple style to list chord hints
\newmenustylesimple*{chordbox}{\hint{\CurrentMenuElement}}[]{blacknwhite}
\renewmenumacro{\chordbox}[,]{chordbox}
\newcommand{\chordlist}[1]{
	\chordbox{#1}
	\vspace{3ex}
}

% Package used to typeset chord marks
\usepackage{stackengine}

% Song environment
\newenvironment{song}[2]
{
	\newpage
	\section[#1 (#2)]{#1 \\{\emph{\large (#2)}}}
}

% Verse environment
\renewenvironment{verse}
{
%	\setlength{\parskip}{-1.5em}
	\setlength{\baselineskip}{2.5em}
}

% Meta-command for chord marks (\chord{Am})
\newcommand{\metachord}[1]{\stackengine{2.3ex}{}{#1}{O}{l}{\quietstack}{T}{\stacktype}}
% Different versions of chord-marks (with bar, bar-less, ...)
\newcommand{\barchord}[1]{\metachord{\rule[-0.2ex]{0.2ex}{0.9em}\,\,#1}}
\newcommand{\chord}[1]{\metachord{\rule[-0.2ex]{0ex}{0.9em}\color{blue}\bfseries#1}}

% Turning off numbering of sections
\setcounter{secnumdepth}{0}

\title{Ukulele minisongbook}  % Title
\date{}                       % Removing date after title
\author{}

\setlength{\parindent}{0em}


\usepackage{listings}
\lstset{basicstyle=\ttfamily\small}

\usepackage{amsmath}

% ----------------------------------------------------------------------------------------
\begin{document}
	
	% Title	
	\maketitle	
	\tableofcontents
	
	
	% Songs
	\begin{song}{Jolene}{Dolly Parton}
		
		\chordlist{Am,C,G,Em}
		
		\textbf{Chorus:} \emph{(at the start and after every verse)}\\
		
		\begin{verse}
			Jo\chord{Am}lene, \quad Jo\chord{C}lene,\quad Jo\chord{G}lene,\quad Jo\chord{Am}lene \quad\chord{Am} \\
			I`m \chord{G}begging of you \chord{G}please don`t take my \chord{Am}man. \quad\chord{Am} \\
			Jo\chord{Am}lene, \quad Jo\chord{C}lene,\quad Jo\chord{G}lene,\quad Jo\chord{Am}lene \quad\chord{Am}\\
			\chord{G}please don`t take him \chord{Em}just because you \chord{Am}can. \quad\chord{Am}\\
		\end{verse}
		
		\begin{verse}
			Your \chord{Am}beauty is be\chord{C}yond compare, \\
			with \chord{G}flaming locks of \chord{Am}auburn hair, \\
			with i\chord{G}vory skin and \chord{Em}eyes of emerald \chord{Am}green. \\
			Your \chord{Am}smile is like a \chord{C}breath of spring, \\
			your \chord{G}voice is soft like \chord{Am}summer rain \\
			and \chord{G}I cannot com\chord{Em}pete with you, Jo\chord{Am}lene. \\
		\end{verse}
		
		\begin{verse}
			He \chord{Am}talks about you \chord{C}in his sleep, \\
			there`s \chord{G}nothing I can \chord{Am}do to keep \\
			from \chord{G}crying when he \chord{Em}calls your name, Jo\chord{Am}lene.  \\
			And \chord{Am}I can easily \chord{C}understand \\
			how \chord{G}you could easily \chord{Am}take my man, \\
			but you \chord{G}don`t know what he \chord{Em}means to me, Jo\chord{Am}lene. \\
		\end{verse}
		
		\begin{verse}
			\chord{Am}You could have your \chord{C}choice of men \\
			but \chord{G}I could never \chord{Am}love again. \\
			\chord{G}He`s the only \chord{Em}one for me, Jo\chord{Am}lene. \\
			I \chord{Am}had to have this \chord{C}talk with you, \\ 
			my \chord{G}happines de\chord{Am}pends on you, \\
			what\chord{G}ever you de\chord{Em}cide to do, Jo\chord{Am}lene.
		\end{verse}
		
	\end{song}
	\begin{song}{One Day}{Matisyahu}
		Intro: C G Am F \\
		
		\begin{verse}
			\chord{C} \quad Sometimes I \chord{G}lay under the \chord{Am}moon, I thank \chord{F}God I'm breathing. \\
			\chord{C} \quad Then I \chord{G}pray don't take me \chord{Am}soon, 'cause I am \chord{F}here for reason. \\
		\end{verse}
		
		\begin{verse}
			\chord{C}Sometimes in my tears I \chord{G}drown, but I \chord{Am}never let it get me \chord{F}down. \\
			So when \chord{C}negativity sur\chord{G}rounds, I \chord{Am}know someday it'll \chord{F}all turn around because \\
		\end{verse}
		
		\begin{verse}
			\chord{C}All my life I've been \chord{G}waiting for, \\
			I've been \chord{Am}praying for, for the \chord{F}people to say \\
			\chord{C}That we don't wanna \chord{G}fight no more, \\
			there'll be \chord{Am}no more wars, and our \chord{F}children will play \\
			\chord{C}One day, \chord{G}one day, \chord{Am}one day,  \chord{F}oh oh oh \\
			\chord{C}One day, \chord{G}one day, \chord{Am}one day, \chord{F} oh oh oh \\
		\end{verse}
		
		\begin{verse}
			It's \chord{C}not a\chord{G}bout win or \chord{Am}lose, 'cause we all \chord{F}lose when they feed on the \\
			\chord{C}Souls of the innocent, \chord{G}blood-drenched pavement,  \\
			\chord{Am}keep on movin' though the \chord{F}waters stay ragin'. \\
			\chord{C}In this \chord{G}maze you could lose your \chord{Am}way, your \chord{F}way \\
			It might \chord{C}drive you crazy, but \chord{G}don't let it phase you no \chord{Am}way, no \chord{F}way. \\
		\end{verse}
		
		Pre-chorus \qquad Chorus \\ \\
		
		\begin{verse}
			One \chord{C}day this all will change, treat \chord{G}people the same, \\
			\chord{Am}stop with the violence,\chord{F} down with the hate \\
			One \chord{C}day we'll all be free, \chord{G}and proud to be  \\
			\chord{Am}under the same sun \chord{F}singing songs of freedom like \\
			
			\chord{C}Wye oh \chord{G} Wye \chord{Am}oh oh oh \chord{F} \\
			\chord{C}Wye oh \chord{G} Wye \chord{Am}oh oh oh \chord{F} \\
		\end{verse} 
		
		Chorus  \\
		
	\end{song}
	\begin{song}{Riptide}{Vance Joy}
		\chordlist{Am,G,C,Fmaj7}
		
		Intro: Am G C C x2 \\ \\
		
		\begin{verse}
			\chord{Am}I was scared of dentis\chord{G}ts and the \chord{C}dark, \chord{C} \\
			\chord{Am}I was scared of pretty\chord{G} girls and star\chord{C}ting conversation\chord{C}s, \\
			\chord{Am}Oh all my fri\chord{G}ends are turning \chord{C}green, \chord{C} \\
			\chord{Am}You're the magicians as\chord{G}sistant in their \chord{C}dreams. \chord{C} 
		\end{verse}
		
		Bridge: \\
		\begin{verse}
			\chord{Am}Ooh, \chord{G}ooh \chord{C}ooh \chord{C} \\
			\chord{Am}Ooh,\chord{G} and \chord{C}they come \chord{C}unstuck 
		\end{verse}
		
		Chorus: \\
		\begin{verse}
			\chord{Am}Lady, \chord{G}running down to the \chord{C}riptide, \\
			\chord{C}taken away to the dark\chord{Am} side, \\
			\chord{G}I wanna be your l\chord{C}eft hand ma\chord{C}n. \\
			\chord{Am}I love you wh\chord{G}en you're singing that son\chord{C}g and, \\
			\chord{C}I got a lump in my throa\chord{Am}t 'cause \\
			\chord{G}you're gonna sing the words wr\chord{C}ong \chord{C}\\  
		\end{verse}
		
		\begin{verse}
			\chord{Am}There's this movie that I t\chord{G}hink you'll li\chord{C}ke, \chord{C} \\
			\chord{Am}this guy decides to qu\chord{G}it his job and hea\chord{C}ds to New York City,\chord{C} \\
			\chord{Am}this cowboy's\chord{G} running from himse\chord{C}lf. \chord{C} \\
			\chord{Am}And she's been living o\chord{G}n the highest shel\chord{C}f \chord{C} 
		\end{verse}
		
		\emph{Bridge} \\
		\emph{Chorus} \\
		Interlude: \\
		\begin{verse}
			\chord{Am}I just wanna, I just wanna kn\chord{G}ow,	\\
			\chord{C}If you're gonna, if you're gonna s\chord{Fmaj7}tay, \\
			\chord{Am}I just gotta, I just gotta kn\chord{G}ow, \\
			\chord{C}I can't have it, I can't have it any \chord{Fmaj7}other way \\
			\chord{Am}I swear she's destin\chord{G}ed for the sc\chord{C}reen, \chord{C} \\
			\chord{Am}Closest thing to Michell\chord{G}e Pfeiffer that you've\chord{C} ever seen, \chord{C}oh 
		\end{verse}
		
		\emph{2 x Chorus} \\
		Ending Chorus: \\
		\begin{verse}
			\chord{Am}Lady, \chord{G}running down to the \chord{C}riptide, \\
			\chord{C}taken away to the dark\chord{Am} side, \\
			\chord{G}I wanna be your l\chord{C}eft hand ma\chord{C}n. \\
			\chord{Am}I love you wh\chord{G}en you're singing that son\chord{C}g and, \\
			\chord{C}I got a lump in my throa\chord{Am}t 'cause \\
			\chord{G}you're gonna sing the words wr\chord{C}ong \\ 
			\chord{C}I got a lump in my thro\chord{Am}at 'cause you're \chord{G}gonna sing the wo\chord{C}rds wrong. \chord{C}
		\end{verse}
	\end{song}
	\begin{song}{Where Have All the Flowers Gone}{Pete Seeger}
	\chordlist{G, Em, C, D}
	
	\begin{verse}
		\chord{G}Where have all the f\chord{Em}lowers gone \\
		\chord{C}Long time \chord{D}passing \\
		\chord{G}Where have all the f\chord{Em}lowers gone \\
		\chord{C}Long time \chord{D}ago \\
		\chord{G}Where have all the f\chord{Em}lowers gone \\
		\chord{C}Girls have picked them \chord{D}every one \\
		\chord{C} When will they \chord{G}ever learn \\
		\chord{C} When will they \chord{D}ever \chord{G}learn \\
	\end{verse}
	
	\begin{verse}
		Where have all the young girls gone \\
		Long time passing \\
		Where have all the young girls gone \\
		Long time ago \\
		Where have all the young girls gone \\
		Taken husbands every one \\
		When will they ever learn \\
		When will they ever learn \\
	\end{verse}
	
	\begin{verse}
		Where have all the young men gone \\
		Long time passing \\
		Where have all the young men gone \\
		Long time ago \\
		Where have all the young men gone \\
		Gone for soldiers every one \\
		When will they ever learn \\
		When will they ever learn \\
	\end{verse}
	
	\begin{verse}
		Where have all the soldiers gone \\
		Long time passing \\
		Where have all the soldiers gone \\
		Long time ago \\
		Where have all the soldiers gone \\
		Gone to graveyards every one \\
		When will they ever learn \\
		When will they ever learn \\
	\end{verse}
	
	\begin{verse}
		Where have all the graveyards gone \\
		Long time passing \\
		Where have all the graveyards gone \\
		Long time ago \\
		Where have all the graveyards gone \\
		Covered with flowers every one \\
		When will we ever learn \\
		When will we ever learn \\
	\end{verse}
	
\end{song}
	\begin{song}{Im On Fire Chords}{Bruce Springsteen}
	\chordlist{G, Em, C, D}
	
	\begin{verse}
		
		\chord{G}Hey little girl, \chord{G}is you daddy home \\
		\chord{G}Did he go and leave you \chord{G}all alone \chord{C}\\
		\chord{C}I’ve got a bad \chord{Em}desire\chord{Em},\qquad \chord{C}Oooh\chord{D} I’m on f\chord{G}ire \chord{G}\\
		
		Tell me now baby is he good to you \\
		can he do to you the things that I do, no more \\
		I can take you higher Oooh, I’m on fire \\
		
		Sometimes it’s like som\chord{C}eone took a knife baby \chord{C}edgy and dull \\
		And put a \chord{Em}six inch valley through the m\chord{D}iddle of my sku\chord{C}ll \qquad\chord{C} \\
		
		At \chord{G}night I wake up with the s\chord{G}heets soaking wet \\
		And a fre\chord{G}ight train running through the mid\chord{G}dle of my head, only yo\chord{C}u \\
		can co\chord{C}ol my \chord{Em}desire\chord{Em},\qquad \chord{C}Oooh\chord{D}, I’m on f\chord{G}ire \chord{G}\\
		
		\chord{C}Oooh\chord{D}, I’m on f\chord{G}ire \\
	\end{verse}
	
	
\end{song}
	\clearpage

\begin{minipage}[t]{0.49\textwidth}
	{\Large \textbf{How to Save a Life}}\\[1ex]
	{\large \textbf{(\emph{The Fray})}}

	\bigskip

	\chordlist{G, D, Em, C}

	\bigskip

	\textbf{Strumming:}\\[1ex]
	\begin{tabular}{@{} c@{}c c@{}c c@{}c c@{}c @{}}
		$ \downarrow $ & & $ \downarrow $&  $ \uparrow $& & $ \uparrow $& $ \downarrow $ & \\
		1 & - & 2 & -& 3 & -& 4& -\\
	\end{tabular}

	\bigskip

	\textbf{Capo:} 3rd fret

\bigskip
\bigskip

\textbf{Intro:}
\begin{lstlisting}[mathescape=true]
G D G D
\end{lstlisting}
\bigskip
\textbf{Chorus:} (\emph{after every verse})
\begin{lstlisting}[mathescape=true]
  C             D                Em
Where did I go wrong, I lost a friend
          G$ ^{1/2\text{ bar}} $             D$ ^{1/2\text{ bar}} $
Somewhere along in the bitterness
 C                   D                Em
And I would have stayed up with you all night
             G$ ^{1/2\text{ bar}} $  D$ ^{1/2\text{ bar}} $   G
Had I known how to save a life
\end{lstlisting}
\bigskip
\textbf{Bridge:} (\emph{after chorus})
\begin{lstlisting}[mathescape=true]
D/F#  Em  D/F#
\end{lstlisting}

\end{minipage}\hfill%
\begin{minipage}[t]{0.49\textwidth}
\textbf{Verse 1:}
\begin{lstlisting}
G                    D/F#
Step one you say we need to talk
    Em                 D/F#             G
He walks you say sit down it's just a talk
             D/F#           Em
He smiles politely back at you
             D/F#              G
You stare politely right on through
               D/F#          Em
Some sort of window to your right
             D/F#              G
As he goes left and you stay right
              D/F#             Em
Between the lines of fear and blame
                   D/F#
And you begin to wonder why you came
\end{lstlisting}
\bigskip
\textbf{Verse 2:}
\begin{lstlisting}
G                 D/F#
Let him know that you know best
       Em           D/F#
Cause after all you do know best
G                  D/F#
  Try to slip past his defense
Em               D/F#
Without granting innocence
G             D/F#             Em
   Lay down a list of what is wrong
             D/F#
The things you've told him all along
G                   D/F#
And pray to God he hears you
    Em             D/F#
And pray to God he hears you
\end{lstlisting}
\bigskip
\textbf{Verse 3:}
\begin{lstlisting}
    G            D/F#
As he begins to raise his voice
Em              D/F#                  G
You lower yours and grant him one last choice
					 D/F#
Drive until you lose the road
    Em                         D/F#
Or break with the ones you've followed
G             D/F#
   He will do one of two things
Em                D/F#
He will admit to everything
G                     D/F#
Or he'll say he's just not the same
     Em             D/F#
And you'll begin to wonder why you came
\end{lstlisting}
\end{minipage}%

	\clearpage
	\input{preprocessing/how-to-save-a-life.txt.tex}
	\clearpage
	\input{preprocessing/bedna-od-whisky.txt.tex}
\end{document}